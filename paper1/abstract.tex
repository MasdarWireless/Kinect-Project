In an average open office building, air conditioning, ventilation and
lighting account for 30 to 40 percent of energy consumption. Nowadays,
most modern conditioning systems in buildings still operate based on
occupancy rather than actual usage. Such operation mode creates
needless conditioning and energy waste. Therefore, in order to achieve
an optimal conditioning state based on  traffic in zones of interest,
we need to know the rate and time of occupancy and intelligently tune
the system according to the  number of actual occupants.  In our
study, we use a people counting software based on a network of
Microsoft Kinect for Windows sensors in order to acquire temporal
occupancy information. The occupancy counter software provides
real-time occupancy data through detection and tracking of people in
the building.  An HVAC management and control system needs to adjust
this data in real-time and measure the local level of comfort.  In
this research, we propose an approach for energy saving which
integrates a real-time occupancy data into building management
systems. This approach leads us to the creation of an occupancy
monitoring and control system which takes into account three elements:
subject mobility, room status (e.g. empty, occupied, crowded) and
actual number of people.  In addition, in order to model the occupancy
data collected, we use the Markov chain principle where a state is a
combination of the statuses of existing zones in the building. Such
state represents the level of energy consumption in real time and a
useful input data for the HVAC system controller.  Here, we
demonstrate that our model, based on data collected by an ensemble of
Kinect sensors, can be integrated with an HVAC control strategy to
achieve substantial energy savings. Through the prediction of future
occupancy level of particular zones of a building, our intelligent
system is able to adjust conditioning parameters gradually to reflect
the predicted changes in time.  