\section{Introduction}
\label{sec:intro}

A 2009 report by the United Nations Environment Program (UNEP SBCI)
\cite{huovila2009buildings} has clearly identified  construction
buildings as responsible of a significant amount of global energy use
and greenhouse gas emissions in both developed and developing
countries. Although most local governments have taken steps through
regulations and policies to reduce energy use and gas emissions, their
efforts have had little impact in the past. In addition,  these
measures are likely to meet some resistance in the future because they
may not serve the economic interests of the many stakeholders involved
in the building sector. Therefore,  there is a need for an innovative
system that implements energy efficiency measures. Such system would
provide locally small energy-reduction opportunities for each of the
millions of buildings across the globe.

At the core of energy consumption in buildings, are Heat Ventilation
And Cooling systems (HVAC) which are mostly designed to operate at
full capacity under the assumption of normal occupancy of rooms at all
time. Although current HVAC systems are equipped with sensors, their
management and control systems ignore the dynamic nature of daily
occupancy of buildings. In addition, they are unable to proactive
adjust to occupants' comfort levels. Understanding	human mobility
and occupancy patterns are key factors in successfully managing energy
in buildings. Building occupancy has been the subject of intensive
studies in the past years. Several approaches using building occupancy
data to improve prediction and simulation of HVAC control have been
proposed. In the same perspective, the main objective of our paper is
to propose an energy-saving model based on occupancy patterns of human
mobility in buildings. This model offers a solution for the management
and control of HVAC systems in smart buildings. The most important
features of the system that implements this model are the following:

\begin{enumerate}
\item  The {\em detection and tracking of people in real time} in the
  building provides accurate occupancy data of an entire building
  divided into several related zones.
\item  A  {\em Occupancy Counter Software} carries out the detection,
  tracking and monitoring process based on multiple Microsoft Kinect
  for Windows (K4W) sensors distributed in strategic locations in the
  building.
\item A {\em prediction of future occupancy} of the building is
  introduced through the use of a Markov chain (MC) which models the
  collected occupancy data.  MC is a suitable because it captures the
  temporal nature of occupancy variation along with inter-room
  correlations and occupant usage.  Unlike most building occupancy
  techniques described in the Related works
  Section~\ref{sec:relatedworks}, our approach implements an occupancy
  counting technique that is based on the Microsoft Kinect for Windows
  (K4W) device.
\end{enumerate}

{\bf Organization:}This paper is organized as follows: In
Section~\ref{sec:background}, we present relevant background
information about human detection and tracking techniques and tools as
well as the most popular occupancy counting and sensing devices. In
particular, we mention the Microsoft Kinect for Windows that is
central to our experimentation. Next, in Section~\ref{sec:design}, we
discuss about the design and implementation of our solution to
energy-saving problem in smart buildings.  Here, we give a detailed
account of the setup of our human mobility tracking and detection
system based on a occupancy counter software using the Kincet for
windows sensor. In Section~\ref{sec:relatedworks}, we expose several
approaches using building occupancy data to improve prediction and
simulation of HVAC control. These approaches use models or techniques
that are related  to our study. Then, in Section~\ref{sec:evaluation},
we displays the results and evaluations of our real world test-bed
conducted in an open office laboratory.  In
Section~\ref{sec:conclusion}, we conclude our paper and propose some
perspectives. 

