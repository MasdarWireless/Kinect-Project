\section{Conclusion}
\label{sec:conclusion}
The original motivation for this study was to provide a fully automated control system for HVAC usage in smart buildings.  In this paper, a new concept of occupancy monitoring in real-time is investigated based on human mobility detection and tracking via occupancy counter software. This is the first study that uses multiple Kinect sensor in building monitoring and management. It has been shown how the Kinect could be applied successfully in the fields of detection and tracking. Kinect is revolutionary technology that enables people to visualize things with highly reliable accuracy. An occupancy model is proposed that displays visualized data and predicts future occupancy through a Markov Model transition matrix. This occupancy model is capable of presenting four different conditions for each zone in the building which refer to the level of occupancy in that zone. Every zone has a state which represents one of the four conditions of the occupancy level {(E)mpty, (F)ew, (A)verage, (C)rowd}.  The model was also designed to alleviate the high cost of energy. Through extensive real time testing of our system, the experimental results have shown that it was able to predict future space occupancy with a satisfactory accuracy while managing incoming online problems.  Based on these results, the system were able to inform us when it was possible to save energy during lower occupancy periods or to turn devices OFF/ON throughout the day. The key challenges in this study were the architectural design of open offices.  Air-conditioning control also required previous knowledge of occupancy in order to meet a certain comfort level which shortened the cyclic control method. We implement and evaluate the monitoring system which was implemented and evaluated in the real-world and we conclude with these observations:
\begin{enumerate}
  \item The occupancy pattern and activity behaviour can be easily derived from human mobility tracking data.  It is possible to notice the working hours and the extra activities that are performed by the occupants throughout this collected data. It is interesting to note that it was possible to create some schedules and patterns.
  \item The nature and structure of open office buildings are difficult to fully monitor and control. Even with the use of assumptions there will be a portion of the data that is lost.
  \item The customized occupancy schedule for each day in the week provides better results than using one unified schedule for the weekday. Each day has its own schedule that must be recorded and should be used to adjust our energy consumption. Rather than assuming full occupancy for a building, a simple monitoring system can run and utilize the usage in real time, and make adjustments accordingly.

\end{enumerate}

We provided the first study that uses multiple Kinect sensor as a sensor in building monitoring and management. Through extensive real time testing of our system, the experimental results showed a potential energy saving around 22.1\% by applying a customized occupancy schedule and customized set point schedule depend on occupancy density. 