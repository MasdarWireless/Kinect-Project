\section{Related Works}
\label{sec:relatedworks}

The purpose of this section is to highlight works related to our approach in analysing human mobility. Building occupancy has been the subject of numerous studies for improving HVAC control. The following subsection gives more details.

\subsection{ HVAC control approaches}
There are a growing number of academic and industrial contributions to HVAC control strategies intended to reduce energy consumption in inhabited buildings. The majority of these contributions rely on occupancy models to produce occupancy simulations of an entire building. These simulations in turn would serve to calculate thermal loads with the intention of correctly provisioning the  HVAC system. There exist other remarkable approaches to HVAC control that start with the simple idea that occupancy can be inferred directly from the analysis of occupants’ movement within the different sections or outside of a building. There is neither requirement nor constraint on the number of occupants and zones. To this end, these approaches like in \cite{buildingoccupancyusingmarkov} implement a Markov chain method to simulate the occupant’s movements. Here, the model is able to do two things with satisfaction: detect each occupant’s location and evaluate the occupancy rate in each zone of the building. In addition, it displays a realistic picture of daily building variation. Others put an emphasis on probability densities instead of making specific predictions. These models make no future projection about the likelihood of presence nor do they mention the actual number of occupants. For example, Page et al in \cite{stochasticsimulation}  model occupancy through a Markov chain. In this model, persons in particular zones of a building are modelled as time series simulating behaviour. In Page’s model, the main feature is the creation of an occupancy probability density on a daily basis.   The works of Liang and Lu deals with the aspects of combining human learning with power control strategies in the design of intelligent and malleable HVAC control strategies \cite{Designofintelligentcomfort}. Here, the authors used a minimum power control strategy which balances and reduces respectively the input power of an HVAC and the energy consumption. They also put into play six parameters : occupancy level, ambient and radiant temperatures, air strength, relative humidity, occupants’ clothing. Then, they designed a human learning strategy based on the Predicted Mean Vote (PMV) model. This allowed them to adjust the user’s comfort zone by learning his/her preferences. Fountain et al. were the first to introduce the idea of comfort zone in the design of control strategies in special occupancy contexts like hotels \cite{Comfortcontrol}. As explained above, comfort zone and human learning strategy have been applied together for thermal comfort control. In addition, Fountain used a Neural Network to circumvent the non-linear aspects of PMV calculation.   With a small number of rules in a Fuzzy System (FS), Arabinda has been able to create in \cite{invertedpendulum} a Neuro-fuzzy controller (NFC) which saves significant computational time. He first designed an FS containing 36 rules of which a few are used for training in a back propagation algorithm. Then, he applied neural network made of three layers of respectively 2, 30 and 1 neurons. This NFC exhibited a noticeable improvement in peak and time for transfer functions in the air supply models in ANF and PID controller. Alcal’a et al. acknowledge that FLCs (Fuzzy Logic Controllers ) are useful for the implementation of expert knowledge and control of HVAC systems. Here, it’s about using linguistic rules and facing the difficulty of the actual knowledge acquisition and elicitation when solving a particular HVAC control problem. They proceeded like with any expert system engineering. That is, a human expert in HVAC control was first extensively interviewed, his practical knowledge is extracted, elicited and then transformed into practical rules that make up an initial Knowledge Base (KB). Only a manageable number of control rules were necessary to partition the system because of the use of an expert knowledge.

The authors designed accurate models to simulate two experimental test buildings. Then the traditional controller was compared with the controller with genetically tuned parameters in the same context buildings during ten days. In the end, the controller studied was implemented and tested. As a result, not only was the level of thermal comfort the same as the traditional physical setting but there was a noticeable decrease in energy consumption by more than 10\%. In the same line of thought, Gacto et al. proposed in [21] an advanced evolutionary Multi-Objective Genetic Algorithm (MOGA) to increase the performance of HVAC system GA tuning of FLCs. Then, others have used an adaptation of MOGA to determine their effectiveness in fast convergence. Besides, an intelligent crossover operator and a GA technique for incest prevention (population diversity without unnecessary crossovers), have been used to improve the algorithm search ability. Using soft computing methods is a popular approach for automatic generation of rule-based fuzzy systems (Fuzzy RB).

\subsection{Building Occupancy Monitoring}
There are several occupancy-based methods that use multiple sources of sensory input. Some suggest that occupancy can be represented using linear regression models. Data gathered for lighting, material loads and occupancy is evaluated with a building walk through survey. A noticeable limitation of this model is its dependence on energy usage to detect the presence of a person. More often, its estimation of occupancy is weak specially when dealing with large groups in a conference room. The same remark is true about conditioning as with energy consumption. Sometimes, the EnergyPlus tool is used to estimate savings. Here a reactive strategy is used in adjusting temperature based on occupancy. In other occasions, door activity detection and PIR sensors for presence detection are used to distinguish the status of a home between occupied, unoccupied, occupants awake or asleep. The estimation of this model does not seem to consider ventilation which is a significant source of energy consumption and ignores daily schedules like an occupant’s activity on a Friday.

In the work titled Occupancy-Based Demand Response [30], the authors introduce an HVAC control strategy to achieve efficient conditioning. It relies on demand response and the real-time monitoring and occupancy prediction. In this occupancy based demand response approach, the authors describe an HVAC control strategy which utilizes a room occupant monitoring system. This system is able to detect in real time, the number of occupants of a room, infer its temperature and level of carbon dioxide (CO2). The most significant contribution of this research is the ability of this system to predict room occupancy. Prediction is necessary because it appears that in general, an HVAC system requires a little time to bring an ambient temperature to a certain level of human comfort according to the American Society of Heating, Refrigeration and Air-Conditioning Engineers. To achieve their goal, the authors  modelled room occupancy as a Markov chain by identifying each status of the room (level of occupancy, vacancy) as a state to which a transition probability is ascribed for moving to the next state (room status). From this model, occupancy includes ventilation and temperature control strategies which are then implemented in an EnergyPlus model. EnergyPlus is building energy simulation software with many innovative simulation features like heat balance-based zone simulation, distributed air flow, thermal comfort, water usage, outdoor ventilation, and solar systems. It is used to optimize design and save on heating, cooling, lighting, ventilation, other energy flows, and water use. For this purpose, it takes charge of parameters like the components of the HVAC system, the level of occupancy, the climate and the construction material.

\subsection{Model Predictive Control}
In contrast to many approaches that focus on general aspects described in the section above, there are a variety of predictive and adaptive models designed for medium size contexts like offices, labs and classrooms. In the contribution titled Network of Sensing, Learning and Prediction Agents  \cite{MamidiChangMaheswaran} the authors introduce a system of multiple adaptive sensor agents whose roles are to detect motion, read CO2, record sound level, ambient light and check door status (open, close). This innovative application called Building-Level Energy Management Systems (BLEMS) is in fact a multi-agent system made of fifty eight multi-modal sensors, scores of teach collaborative agents that adapt to occupants’ particular needs. In addition, it contains 74 actuators related to the building’s HVAC areas and two units for handling central air. In practice, patterns of occupants’ activity are acquired through observation. Then, HVAC operation is optimized in response to the occupant models. On the other hand, by creating an agent model of each occupant, it is possible to predict room occupancy rate. The purpose of this system is to create an appropriate balance between energy preservation and occupants’ comfort through the use of machine learning techniques in areas that are likely to be occupied. This system has been successfully deployed and able to estimate occupancy with a 95\% accuracy rate. The deployment setting is the premises of the University of Southern California (USC).

With OBSERVE, Erickson et al. show in \cite{EricksonCarreiraPerpinan} how to use a wireless sensor network to collect real time occupancy data and use it to create occupancy models. Such models may be included in a building system for control strategies. With occupancy model predictions drawn from a sensor network-based control strategy, the authors confirm that they have achieved 42\% annual energy saving without compromising the American Society of Heating, Refrigerating and Air-Conditioning (ASHRAE) comfort standards.  Xiang et al. present in the article Smart Personalized Office Thermal Control System (SPOT) \cite{SPOT} a smart personal thermal comfort system for use in an office environment. The role of this system called SPOT (Smart Personalized Office Thermal) is to find an acceptable balance between energy consumption and personal thermal comfort in an office environment. This is a reactive control strategy that takes into account real-time occupancy and personal thermal comfort. It rests on an original model of personal thermal comfort known as the Predicted Personal Vote (PPV) model. SPOT makes use of a collection of sensors including Microsoft Kinect to evaluate six parameters known to define human comfort: clothing, air speed, humidity, radiant temperature, and air temperature and activity level. These parameters are essential to the PPV model which guides SPOT in controlling heating and cooling parts that maintain comfort. In spite of its appealing features and effective results, the SPOT model of HVAC control has several inherent limitations related to the designers’ initial restricting assumptions like  spaces are thermally isolated from one another, confinement to a personal space, long lasting calibration process and the excessive cost of the systems.

